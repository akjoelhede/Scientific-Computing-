\documentclass[a4paper,12pkt]{report}
\usepackage[utf8]{inputenc} %Specialtegn
\usepackage{amsmath} %Matricer og Mat symboler
\usepackage{graphicx} %Billeder og Figurer
\usepackage{setspace} %Mellemrum mellem linjer og paragraffer
\usepackage{amssymb} %Mat symboler
\usepackage[danish]{babel} %Stavekontrol
\usepackage{fancyhdr} %Headers
\usepackage{csquotes} %Citeringer over flere dokumenter
\usepackage[a4paper, total={6in, 8in}]{geometry} %Bredere Marginer
\usepackage[backend=biber]{biblatex} %Gør det muligt at have litteraturliste med flere subfiles
\addbibresource{Litteratur.bib} %Litteraturliste
\bibliography{Litteratur.bib} %Litteraturliste
\usepackage[colorlinks]{hyperref} %Hyperlinks i PDF
\title{Scienfic Computing Project 1 }
\author{Anders Kjølhede}
\date{September 2022}

\begin{document}

\pagestyle{fancy}
\fancyhf{}
\rhead{Scienfic Computing}
\lhead{nr.1}
\rfoot{side \thepage}

\maketitle

\onehalfspacing %Linjeafstand
\setlength{\jot}{10pt} %Mellemrum ved ligninger

\section*{Background Theory}

The polarizability for a given frequency $\omega$ is obtained as a scalar product of two vectors \textbf{z} and \textbf{x}.

\begin{equation}
    \alpha(\omega) = z^Tx
\end{equation}

\textbf{z} is obtained from the schrödinger equation and \textbf{x} is the solution to the following system of linear equation:

\begin{equation}
    (E-\omega S)x = z
\end{equation}

E and S are square matrices and $\omega$ is the frequency of the incoming light. E, S and z are visualized as follows:

\begin{equation}
    E = \begin{bmatrix}
        A & B \\
        B & A
    \end{bmatrix}
    \hspace{1cm}
    S = \begin{bmatrix}
        I & 0 \\
        0 & -I
    \end{bmatrix}
    \hspace{1cm}
    z = \begin{bmatrix}
        y \\
        -y
    \end{bmatrix}
\end{equation}

\section*{Question A}

By computing the max-norm and the inverse of the given matrix we can calculate the condition number for the individual frequencies by using the following relation:

\begin{equation}
    cond_{\infty}(M) = ||M||_{\infty}||M^{-1}||_{\infty}
\end{equation}



\begin{table}[h!]
    \centering
    \begin{tabular}{l  c  c  c }
    \hline
     Frequencies & Condition Number \\ \hline
     0.800 & 327.8167 \\
     1.146 & 152679.2687\\
     1.400 & 227.1944
    \end{tabular}
    \caption{}
    \label{Confusion matrix scores}
\end{table}

\section*{Question B}

For the three frequencies we can also determine a bound on the relative forward error in the max-norm:

\begin{equation}
    \frac{||\delta x||_{\infty}}{||\hat{x}||_{\infty}} \leq cond_{\infty}(E-\omega S) \frac{||\delta \omega S||_{\infty}}{||E-\omega S||_{\infty}}
\end{equation}

\begin{table}[h!]
    \centering
    \begin{tabular}{l  c  c  c }
    \hline
     Frequencies & Bound \\ \hline
     0.800 & 0.0052 \\
     1.146 & 2.4050\\
     1.400 & 0.0035
    \end{tabular}
    \caption{}
\end{table}

\newpage

\section*{C}

\end{document}